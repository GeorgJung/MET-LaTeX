\documentclass[pdftex,quiz,formula]{newcsen}

\Topic{Topic of this assessment}
\Date{December 21, 2012}
\Logo{FExtinguish}
%\Assessmenttype{Practical Assignment}
\Version{3}
\Scratchpaper{2}

\begin{document}

\begin{exercise}{5}{Theme}{Name}
  Test 1: With theme and name, 5 Marks
  \begin{solution}
    Pseudo solution. Should appear when option ``solution'' is set.
  \end{solution}
\end{exercise}

\begin{figure}[ht]
  \centering
  \vspace*{3cm}
  \caption{Ganz in weiß}
  \label{fig:test}
\end{figure}

\begin{exercise}{3}{}{}
  Test 2: No theme, no name, 3 Marks submarks are added for checking
  \AddFormula[Formulaname]{Formula with name}
  \AddFormula{$\sum_{k\leq n}$}
  \begin{enumerate}
  \item Part 1\Mark{1}
%  \item Part 2\Mark{1}
  \item Part 3\Mark{1}
  \item Part 4 (Bonus)\Bonus{1}
  \end{enumerate}
\end{exercise}

\begin{exercise}{6}{}{}
  Test 3: No theme, no name, 6 Marks
\end{exercise}

\begin{exercise}{}{}{}
  Test 4: No theme, no name, no Marks. Should produce clear output
  when marking is active
\end{exercise}

\begin{bonusexercise}{13}{Bonus}{Bonusexercise}
  Test 5: Bonusexercise with theme and name and 13 marks. Should be
  indistinguishable from normal exercise if marking is off
\end{bonusexercise}

\begin{bonusexercise}{}{No}{Marks}
  Test 6: Bonusexercise with theme and name and no marks. Should be
  indistinguishable from normal exercise if marking is off, should
  produce clear warning if marking is active.
\end{bonusexercise}

\end{document}

%%% Local Variables:
%%% mode: latex
%%% TeX-master: t
%%% End:
